\documentclass[english]{report}


\setlength {\marginparwidth }{2cm} 
\usepackage{todonotes}

\usepackage[perpage,para,symbol]{footmisc}

\hyphenpenalty=15000 
\tolerance=1000

\usepackage{tikz}
\usetikzlibrary{arrows,decorations.pathmorphing,backgrounds,fit,positioning,calc,shapes}
\usepackage{pgfmath}
\usepackage{rotating}
\usepackage{array}	
\usepackage{graphicx}
\usepackage{float}	
\usepackage{mdwlist}
\usepackage{setspace}
\usepackage{listings}
\usepackage{bytefield}
\usepackage{tabularx}
\usepackage{multirow}	       
\usepackage{caption} 
\captionsetup[table]{skip=10pt}

\usepackage{url}               
\usepackage{hyperref}
\usepackage[all]{hypcap}	
\usepackage{titlesec}
\setcounter{secnumdepth}{4}
\titleformat{\paragraph}
{\normalfont\normalsize\bfseries}{\theparagraph}{1em}{}
\titlespacing*{\paragraph}
{0pt}{3.25ex plus 1ex minus .2ex}{1.5ex plus .2ex}
\hypersetup{colorlinks,breaklinks,
            linkcolor=darkblue,urlcolor=darkblue,
            anchorcolor=darkblue,citecolor=darkblue}


\definecolor{darkblue}{rgb}{0.0,0.0,0.3} 
\definecolor{darkred}{rgb}{0.4,0.0,0.0}
\definecolor{red}{rgb}{0.7,0.0,0.0}
\definecolor{lightgrey}{rgb}{0.8,0.8,0.8} 
\definecolor{grey}{rgb}{0.6,0.6,0.6}
\definecolor{darkgrey}{rgb}{0.4,0.4,0.4}
\definecolor{aqua}{rgb}{0.0, 1.0, 1.0}
\definecolor{dkgreen}{rgb}{0,0.6,0}
\definecolor{gray}{rgb}{0.5,0.5,0.5}
\definecolor{mauve}{rgb}{0.58,0,0.82}

\lstset{
  language=C,
  showstringspaces=false,
  columns=flexible,
  basicstyle={\small\ttfamily},
  numbers=none,
  numberstyle=\tiny\color{gray},
  keywordstyle=\color{blue},
  commentstyle=\color{dkgreen},
  stringstyle=\color{mauve},
  breaklines=true,
  breakatwhitespace=true,
  tabsize=3
}
\usepackage{listings}
\PassOptionsToPackage{USenglish,english}{babel} 
\usepackage{csquotes}
\usepackage[USenglish,english]{babel}
\usepackage[acronym, section=section, nonumberlist, nomain, nopostdot]{glossaries}
\makeglossaries
 
\makeglossaries
\newcommand{\colorbitbox}[3]{%
	\rlap{\bitbox{#2}{\color{#1}\rule{\width}{\height}}}%
	\bitbox{#2}{#3}}

\begin{document}

\title{Homework 1}
\author{Calin Capitanu}

\maketitle

\clearpage
\chapter{Science as Falsification}

\section{Followers Conventional Twist}
One of the conclusions of the Popper's essay is that ``Some genuinely testable theories, when found to be false, are still upheld by their admirers — for example by introducing
ad hoc some auxiliary assumption, or by reinterpreting the theory ad hoc in such a way that it escapes refutation. Such a
procedure is always possible, but it rescues the theory from refutation only at the price of destroying, or at least lowering, its
scientific status.'' \cite{popper}. I chose this conclusion as it seems important are highly relevant in today's society, where people that gain trust become trusted whatsoever and supported in actions that one would generally not support, but since they are supported by their models, they thus support them too.
\\\\
An example that I found in Popper's essay that supports this conclusion is the one of the Marxist theory of history. This example directly supports the previously formulated conclusion, as it shows that Marx's followers, once saw that the theory was testable and falsified, they started re-itnerepreting the theory and the evidence, just to have the theory still hold. Changing different ``parameters'' of the theory, the evidence or the testability of such, in fact does change the theory in itself, however, his followers wanted to rescue it. The price of this is that when they did it, they gave it a ``conventional twist'', which is what Popper calls it. This in fact destroys the strategy, instead. It proves even further that the theory did not hold from the beginning, and it even deeper accentuates the initial falsification. \cite[p.~3]{popper}
\section{Tabula Rasa}
One interesting theory that dates from 1689 is the Blank State theory (also known as Tabula rasa). This theory states that a newborn has no previous knowledge, but that everything is gained through experiences and learning. This theory is really interesting and has its roots in genetics science. It was initially developed by John Locke in the ``Essay Concerning Human Understanding''. In this, he described the mind of a newborn as being as a ``blank slate'' (tabula rasa).
\\\\
However, with science evolution and further theory studies, this theory has later been proven to be falsified, as more genetic science and experimenting, as well as theories have been developed. It is interesting to see that the theory is, however already falsified, still believed in to the present. Robert Duschinsky, in the article ``Tabula Rasa and Human Nature'' \cite{tabula-rasa} took a deeper look in its meaning and how the theory is still present in human beliefs today, however it is already ``debunked''.

\chapter{What is truth?}

We all know that everything is relative. Such, the truth can be relative in this sense. Moreover, when something is true, it can be true from different perspectives, with different types of reasoning.
\\
Below I will take the list of 9 statements and analyze them from the perspective of truth

\begin{enumerate}
  
\item The program statement while (true) {} gives an infinite loop.

  I would argue that this statement is true. The way it is, it is true, however, slight changes to the problem description, and the lack of intuition would make this possibly false. That is, if the programming language of choice is one that handles infinite loops with care, this could be considered false. But from the perspective of pseudo code, this statement is true.
\\
  The type of truth that I considered this is coherence truth, since it can be linked to other true statements, from documentations and definitions of programming language theory. It can be also argued that it is an intuitive truth, but that is based on experience and study.
  
\item Mergesort has complexity O(n log n).

  This statement is also true, which is something that has been proven through mathematics. This one truth I will argue as still coherence truth, since it can be logically deducted from other true statements, such as the mathematical form of the algorithm.

\item Apple suffered losses in the consumer market last year.

  This statement is generally true, if the point of analysis is the whole consumer market. However, if we restrict the margins to some specific country, we can say that this statement is neither true, nor false, or even false, accordingly. Since the formulation is vast, the whole consumer market is taken into account, thus the conclusion that it would be true. This truth is taken from real statistics, shown by third parties.
  \\
  The truth is a correspondence truth, since it can be that it corresponds to reality. The statistics that correspond to reality show the decrease in the market share of Apple for last year, thus that being true, this corresponds to that truth. 

\item Comments make it easier to modify programs.

  This statement is neither true, nor false. It is a subjective perspective on how comments make modifying programs easier or not. At the same time, statistically, this statement has been proved to be true. Personally, I think this statement is true, since I believe it is true.
  \\
  According to the above, I will put this statement into ``Intuitive Truth'', since according to my belief, this statement is true, but it can not be generally true, since for others it is not true. That is, this statement is true for me.

\item Agile development provides greater job satisfaction.

  This statement is also a subjective one, thus making it hard to define whether or not it is true. However, for me, this statement is true, and generally speaking, it is for the majority, as statistics confirmed.
  \\\\
  This statement can be categorized as: Correspondence Truth and Intuitive Truth at the same time:
  \\
  First of all, it is a correspondence truth since it corresponds to the reality of other statistical results. The truth of those results is not be be argued, thus, basing this statement on those, it is also true.
  \\
  On the other hand, it is also an intuitive truth, since this is proven for me personally to have worked this way, thus I intuitively consider this to be true.
  
\item Two doses of the mRNA Covid-19 vaccine BNT162b2 give a 95\% protection for adults.

  This statement is not true, since it does not generally hold. This is a statistical result, but that is, two doses can give more than 95\% protection, or even less. If the statement would have been formulated such that the two doses give ``on average 95\%'', then it would have been true, but I don't consider it true in this way. The article from YaleMedicine \cite{covid} does talk more detailed about the percentages of severe case prevention and how it was initially rated at 91.3\%. This proves (by intuition) that the sample space matters a lot. That is, again, it gives ``on average 95\%'' rate of protection.
  
\item P is a strict subset of NP.

  This is a generally true statement, since there are theories and proofs that show this. This is a mathematical conclusion after proven theories.
  \\
  This truth I will consider as coherence truth, since it is logically related and deducted from other mathematical truths, theorms and definitions. \cite{pnp}
  
\item This statement is true.

  I think that this statement is neither true, nor false. Since it is taken out of context, the only type of truth it could be included in is ``intuitive truth''. There is nothing that can prove that this statement is true, nor that it is false. Since this statement can be both true and false, or neither, or only one, I will categorize this one as being neither.
  \\
  Generally speaking, however it is different from the following one, since considering this statement to be true, would not be a paradox, nothing would contradict.
  
\item This statement is false. (Note - interpret this as "Statement 9 in the second part of HW1 is false")

  Similarly to the previous example, this statement can not be considered true or false. That is, this statement is a paradox in itself. If this statement is to be considered false, than it indeed is ``true'' that this statement is false. One can contradict the other by saying the same.
  \\
  If one would consider this statement true, than the actual meaning of the statement is false, since the statement assumes that it is false, and the other way similarly.

  
\end{enumerate}



\bibliographystyle{myIEEEtran}
\renewcommand{\bibname}{References}
\addcontentsline{toc}{chapter}{References}
\bibliography{references}

\end{document}
